\abstract{
Recent attention for reduced carbon emissions has pushed transit authorities to adopt battery electric buses (BEBs). One challenge experienced by BEB users is extended charge times, which create logistical challenges and may force BEBs to charge when energy is more expensive. Furthermore, BEB charging leads to high power demands, which can significantly increase monthly power costs and may push electrical infrastructure beyond its present capacity, requiring expensive upgrades.  This work presents a novel method for minimizing the monthly cost of BEB charging while meeting bus route constraints. This method extends previous work by incorporating a more novel cost model, effects from uncontrolled loads, differences between daytime and overnight charging, and variable rate charging. A graph-based network-flow framework, represented by a mixed integer linear program, encodes the charging action space, physical bus constraints, and battery state of charge dynamics. Results for three scenarios are considered: uncontested charging, which uses equal numbers of buses and chargers, contested charging, which has more buses than chargers, and variable charge rates. Among other findings, we show that BEBs can be added to the fleet without raising the peak power demand for only the cost of the energy, suggesting that conversion to electrified transit is possible without upgrading power delivery infrastructure. 
}
\keyword{Battery Electric Buses, Cost Minimization, Multi-Rate Charging, Mixed Integer Linear Program}




