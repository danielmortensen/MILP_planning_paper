\section{Literature Review}
\par This section summarizes prior work related to the charge problem and includes discussion on battery charging and managing runtime costs. The final subsection discusses the contributions of this paper, and how they relate to prior methods.
\subsection{Battery Charging}
The refueling process for BEBs takes longer then the process for diesal and CNG buses \cite{wei_optimizing_2018}. A diesal or CNG engine can refuel in several minutes but an electric bus may require up to several hours, making the extended charge time a primary deterrent in converting to BEBs.
\par To circumvent long refuel times, \cite{xian_zhang_optimal_2016} and \cite{jain_battery_2020} propose an approach which replaces batteries when the state of charge is low. The exchange would replace the current battery with one that was fully charged and return spent batteries to recharge afterward. Exchanging batteries would reduce down time, but is non-trivial because the replacement requires specialized tools and/or automation.  
\par Another alternative is to inductively charge buses as they move about. Dynamic charging would simplify logistics because it eliminates the need to stop and recharge. Both \cite{balde_electric_2019} and \cite{jeong_automatic_2018} propose methods that inductively charge BEBs using specialized hardware in the road. A technique developed by \cite{csonka_optimization_2021} provides a planning framework for charger placement that maximizes the benefits of dynamic charging. Unfortunately, inductive charging requires infrastructure which may not be available and is costly and disruptive.
\par Refueling BEBs in a charge station allows buses to charge in the traditional sense and only requires an intelligent charge schedule. Following a charge schedule requires minimal infrastructure and utilizes native charge ports in the BEBs with no need for additional tools or automation. Planning algorithms use foreknowledge of the runtime environment and battery dynamics to identify when, and to which buses chargers should connect. Planning algorithms discussed in this review are considered on a scale from `reactive' to `global', where reactive methods focuse only on the present, and global techniques use knowledge about the operating environment to form a plan.
\par Because reactive planning tend to focus on presential circumstances, they require minimal knowledge of the operational environment, making them extremely versatile.  Methods of this type are both computationally efficient and adapt to many use cases.  One such example is illustrated in \cite{cheng_smart_2020}, which splits the total power draw between the grid and an external battery to regulate the instantaneous load. 
\par Reactive algorithms can be enhanced by encoding details for future events to improve decision making. If only event details within a finite horizon are used, the algorithm becomes a hybrid, containing elements from both reactive and global techniques. For example, \cite{bagherinezhad_spatio-temporal_2020} describes a technique for optimizing a charging schedule up to a scheduling horizon. Changing the horizon adjusts both the scope and computational complexity of the solution. In stochastic environments, a smaller window would be beneficial as charge schedules must be frequently recomputed, whereas in more stable circumstances, a larger window size would yield higher performance. 
\par When an algorithm includes information over all time periods, it becomes global. Because global algorithms assume complete foreknowledge of future events, they provide globally optimal plans and give the highest performance. The authors of \cite{jahic_preemptive_2019} provide a technique which assumes foreknowledge of the current grid use. The grid schedule is encoded in the algorithm to inform optimal charging periods. Both \cite{whitaker_network_2021} and \cite{el-taweel_incorporation_2019} assume that bus schedules are known a priori and use this knowledge to stagger charge times and meet operational constraints. 
\subsection{Cost Optimization}
In addition to charge constraints, This paper focuses on minimizing the cost associated with charging and minimizes over energy charges for on and off-peak, power (or demand) charges for on and off-peak, and facilities charges \cite{noauthor_rocky_nodate}. Prior work has dealt with charge costs in various ways.  
\par The authors in \cite{gao_charging_2019} propose a method to forcast power use. Work done by \cite{qin_numerical_2016} propose a method which reduces the demand charge by using power forcasts \cite{gao_charging_2019} to plan charge times.  When forcasting is not possible, both \cite{ojer_development_2020} and \cite{cheng_smart_2020} propose methods that decreases power demand by observing the load and drawing additional power from an on-site battery. Additionally, \cite{el-taweel_incorporation_2019} minimized over on/off peak energy as part of their work.
\subsection{Contributions}
This work develops a comprehensive charge schedule planning framework which extends the planner proposed by \cite{whitaker_network_2021} to include multi-rate charging, uncontrolled loads, night/day charging, and a realistic rate schedule in \cite{noauthor_rocky_nodate}. Our method formulates the bus charge problem as a Mixed Integer Linear Program (MILP) and is unique because the objective function is the cost for the transit authority and includes charges for on-peak and off-peak energy use, on-peak and off-peak demand, and overall, or facilities, demand. The proposed framework handles contention for charging resources in a globally optimal manner which guarantees charger availability even when chargers are scarse.
\par Prior work has also made assumptions for night time charge behavior. This work eliminates the need for such by including night charging in the charge schedule. The modeling of night and day charging also includes their respective operational constraints such as charge rates, bus availability, and the number of available chargers.
\par Our work also seeks to understand how variable rate, as compared to single rate, charging affects the cost optimality. necessary? with the addition of variable charge rates and a more accurate representation of battery charging dynamics. 
\par The final contribution is recognizing that our framework enables transit authorities and power providers to come together and find a mutually beneficial solution that predicts montly costs for transit authorities and infrastructure demand for power providers.

