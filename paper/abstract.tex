\begin{abstract}
Recent attention to reducing carbon emissions have pushed transit authorities to adopt battery electric buses (BEBs). Two challenges associated BEB are extended charge times, which create logistical challenges and may force BEBs to charge when energy is more expensive. Furthermore, BEB charging leads to high power demands, which can significantly increase monthly power costs and may push electricity distribution circuits beyond their present capacity, revealing hidden costs of BEB adoption. This work presents a comprehensive method for minimizing the monthly cost incurred by charging BEBs while meeting bus route constraints, accounting for uncontrolled loads and both daytime and overnight charging, explores multiple charge rates, and minimizes a realistic cost model.  A graph-based network-flow framework encodes the charging action space, physical constraints of buses, and represents the dynamics of bus battery state of charge.  A mixed integer linear program is derived from the graph.  Optimal charging schedules are explored in three scenarios: uncontested charging (equal numbers of buses and chargers), contested charging (more buses than chargers), and variable rate comparisons.  Among other findings, we show that BEBs can be added to the fleet for only the cost of the energy to charge them but without raising the peak power demand, suggesting that conversion to electrified transit without expensive upgrades to the electrical infrastructure.
\end{abstract}
\begin{IEEEkeywords}
	Battery Electric Buses, Cost Minimization, Multi-Rate Charging, Mixed Integer Linear Program
\end{IEEEkeywords}




