\section{Objective Function}
The objective function in this work models the rate schedule used in \cite{noauthor_rocky_nodate}, where the result is modeled as the monthly charge a transit authority might receive from the power provider. The objective function includes charges for energy, power, and facility use and implements both on and off-peak rates.
\subsection{Energy}
\par Energy charges are given per Kilowatt-hour of energy consumed and include power from external loads and bus chargers. Let $\mathbf{p}$ be the average external power used at each timestep, where $\mathbf{p}_i$ is the average power draw between $t_j$ and $t_{j + 1}$. The energy consumed by external loads from $t_j$ to $t_{j+1}$ is computed as 
\begin{equation}
	e_{l_j} = \mathbf{p}_i \times \Delta_t,
\end{equation}
where $\Delta_t$ is the change in time from $t_j$ to $t_{j+1}$ in hours. The energy consumed by bus chargers for the same interval is computed as  
\begin{equation}
	e_{b_j} = \sum_{k\in t}g_{i,k,l},	
\end{equation}
which sums all changes in bus SOC between $t_j$ and $t_{j+1}$.
The total energy is computed as 
\begin{equation}\label{eqn:energy1}
	e_j = e_{l_j} + e_{b_j}
\end{equation}
The equation for \ref{eqn:energy1} can be written in standard form as 
\begin{equation}\label{eqn:energy2}
	\begin{aligned}
		e_j -\sum_{k\in t}g_{i,k,l} &= p_i \times \Delta_t \\
		\begin{bmatrix} 1_{e_j} & -1_{g_1} & \hdots -1_{g_n} \end{bmatrix} \begin{bmatrix}e_j \\ g_1 \\ \vdots \\ g_n \end{bmatrix} &= p_i \times \Delta_t
	\end{aligned}
\end{equation}
Equation \ref{eqn:energy2} can be modified to reflect the energy consumed in an arbitrary time period, $T$, by including the corresponding values for $g$ and $p$ as
\begin{equation}\label{eqn:energy3}
	\begin{aligned}
	e_j -\sum_{k\in T}g_{i,k,l} &= \left ( \sum_{i\in T}p_i \right ) \times \Delta_t \\
		\begin{bmatrix} 1_{e_j} & -1_{g_1} & \hdots -1_{g_n} \end{bmatrix} \begin{bmatrix}e_j \\ g_1 \\ \vdots \\ g_n \end{bmatrix} &= e_{p_j}.
	\end{aligned}
\end{equation}
For multiple time periods, the constraint can be expanded in matrix form, where row $i$ corresponds to the periods of time in $T_i$. Furthermore, by including $e_j$ in the solution vector $\mathbf{y}$ and zero-padding appropriately, the expanded form of equation \ref{eqn:energy3} can be written as  
\begin{equation}
	E\mathbf{y} = \mathbf{e}_p,
\end{equation}
where row $i$ in $E$ reflects equation \ref{eqn:energy3} for the time intervals in $T_i$, and $\mathbf{e}_{p_i}$ contains the energy from external loads during $T_i$.
The day is divided into on and off-peak times.  on-peak times reflect periods where high power use is common.
\subsection{Power}
Power charges are computed for the maximum average power draw, where the average is computed over a 15 minute sliding window. The average power can be computed as the energy in the window divided by the window length in hours. In this case, a 15 minute window equates to a quarter hour. Let $\bar{p}_j$ be the average power from $j - 15$ to $j$. Equation \ref{eqn:energy3} can be adapted to compute the average power as
\begin{equation}\label{eqn:power1} 
	\begin{aligned}
		\bar{p}_j - \left ( \sum_{k\in T_j}g_{i,k,l} \right )/4 &= \left ( \sum_{i\in T_j}p_i \right ) \times \frac{\Delta_t}{4} \\
		\begin{bmatrix} 1_{\bar{p}_j} & -\frac{1_{g_1}}{4} & \hdots -\frac{1_{g_n}}{4} \end{bmatrix} \begin{bmatrix}e_j \\ g_1 \\ \vdots \\ g_n \end{bmatrix} &= \frac{p_T \times \Delta_t}{4}.
	\end{aligned}
\end{equation}
Equation \ref{eqn:power1} can further be expanded and zero padded to compute the average power at each time, $t_i$ by applying equation \ref{eqn:power1} to the corresponding window as
\begin{equation}\label{eqn:power2}
	P\mathbf{y} = \mathbf{p}.	
\end{equation}
The maximum average power, denoted $\hat{p}$, is greater than or equal to each average power computed in equation \ref{eqn:power2}.  This yields an additional set of inequality constraints 
\begin{equation}\label{eqn:cPower1}
	\begin{aligned}
		 \begin{bmatrix} 
			-1_{\hat{p}} & 1_{\bar{p}_0} & 0             & \hdots & 0 \\ 
	        	-1_{\hat{p}} & 0       & 1_{\bar{p}_1} & \hdots & 0\\
			-1_{\hat{p}} & 0       & 0 & \hdots    & 1_{\bar{p}_j} \\
		 \end{bmatrix}\mathbf{y} &\le \mathbf{0} \\ 
		 P_{\text{max}}\mathbf{y} &\le \mathbf{0}.
	\end{aligned}
\end{equation}
Because the max average power is minimizied in the objective function, the value for $\hat{p}_{\text{max}}$ will be forced down to the value of the greatest average power computed in equation \ref{eqn:power2}, and accurately reflect the maximum average power.
\subsection{On/Off Peak Rates}
Power providers also divide each day into on and off-peak periods during which different rates are applied for both energy and power charges. Let $H$ and $L$ be the respective sets of all time indices in on and off peak periods. The cost of energy during on-peak hours can be expressed as 
\begin{equation}\label{eqn:cEnergy1}
	\begin{aligned}
		c_{\text{energy}_H} &= \left ( \sum_{j\in H} e_j \right ) r_{e_\text{on}} \\
		&= \begin{bmatrix}r_{e_1} & 0 & \hdots & 0 & r_{e_4} & \hdots & 0 \end{bmatrix}\mathbf{y} \\
		&= \mathbf{r}_{e_\text{on}}^T\mathbf{y},
	\end{aligned}
\end{equation}
where $\mathbf{r}_{e_\text{on}}$ contains the value of $r_{e_\text{on}}$ at the index corresponding to $e_j$ in $\mathbf{y} \ \forall j \in H$. A similar formulation can be used to describe the cost of energy consumed during off-peak hours.  
\par An on-peak rate also applies to charges for power. Equation \ref{eqn:cPower1} can be adapted to only include rows that correspond to average power values during on-peak hours such that
\begin{equation}\label{eqn:cPOnPeak} 
	\begin{aligned}
		 \begin{bmatrix} 
			 -1_{\hat{p}_\text{on}} & 1_{\bar{p}_0} & 0             & \hdots & 0 \\ 
			 -1_{\hat{p}_\text{on}} & 0       & 1_{\bar{p}_1} & \hdots & 0\\
			 -1_{\hat{p}_\text{on}} & 0       & 0 & \hdots    & 1_{\bar{p}_j} \\
		 \end{bmatrix}\mathbf{y} &\le \mathbf{0} \\ 
		 P_{\text{on}}\mathbf{y} &\le \mathbf{0}.  
	\end{aligned}
\end{equation}
Similarly, the off-peak max average power can be computed as  
\begin{equation}\label{eqn:cPOffPeak}
	\begin{aligned}
		 \begin{bmatrix} 
			 -1_{\hat{p}_\text{off}} & 1_{\bar{p}_0} & 0             & \hdots & 0 \\ 
			 -1_{\hat{p}_\text{off}} & 0       & 1_{\bar{p}_1} & \hdots & 0\\
			 -1_{\hat{p}_\text{off}} & 0       & 0 & \hdots    & 1_{\bar{p}_j} \\
		 \end{bmatrix}\mathbf{y} &\le \mathbf{0} \\ 
		 P_{\text{off}}\mathbf{y} &\le \mathbf{0},  
	\end{aligned}
\end{equation}
where each row corresponds to $\bar{p}_j \ \forall j \in L$.
\par Many power providers also include a facilities charge.  The facilities charge is billed for each Kw of the maximum average power in \textit{both} on and off-peak intervals. The total max average power is calculated using equation \ref{eqn:cPower1}.
\par The total power charge can be computed as the sum of the on-peak, off-peak, and facilities charges as 
\begin{equation}
	\begin{aligned}
		c_\text{power} &= \begin{bmatrix}r_{\hat{p}_{\text{on}}} & \hdots & 0 & r_{\hat{p}_{\text{off}}} & \hdots & r_{\hat{p}_{\text{facilities}}} \end{bmatrix}\mathbf{y} \\
			&= \mathbf{r}_{\hat{p}}^T\mathbf{y}
	\end{aligned}
\end{equation}

\subsection{Objective Function}
The objective function combines the cost of energy and power, where the on-peak and off-peak energy is combined as 
\begin{equation}\label{eqn:cEnergy2}
	\begin{aligned}
		c_{\text{energy}} &= \mathbf{r}_{e_\text{on}}^T\mathbf{y} + \mathbf{r}_{e_\text{off}}^T\mathbf{y} \\
		&=\left ( \mathbf{r}_{e_\text{on}} + \mathbf{r}_{e_\text{off}} \right )^T \mathbf{y} \\
		&= \mathbf{r}_e^T\mathbf{y}.
	\end{aligned}
\end{equation}
The combined expression is given as 
\begin{equation}\label{eqn:objective}
	\begin{aligned}
		c_{\text{total}} &= c_{\text{power}} + c_{\text{energy}} \\ 
				 &= \mathbf{r}_e^T\mathbf{y} + \mathbf{r}_{\hat{p}}^T\mathbf{y} \\
				 &= \left ( \mathbf{r}_e + \mathbf{r}_{\hat{p}} \right )^T\mathbf{y} \\
				 &= \mathbf{r}^T\mathbf{y}.
	\end{aligned}
\end{equation}
\par Equation \ref{eqn:objective} is used as the objective function in a mixed integer linear program of the form
\begin{equation}
	\begin{aligned}
		& \underset{\mathbf{y}}{\scalebox{1}{\text{min}}} \ \mathbf{r}^T\mathbf{y} \\
		\text{subject to}& \ \ \ \  C_{\text{eq}}\mathbf{y} = \mathbf{c}_{\text{eq}}, \ C_{\text{ineq}}\mathbf{y} \le \mathbf{c}_{\text{ineq}},
	\end{aligned}
\end{equation}
Where $C_{\text{eq}}, \mathbf{c}_{\text{eq}}, C_{\text{ineq}}, and \mathbf{c}_{text{ineq}}$ are formed by stacking the equality and inequality constraints from equations \ref{eqn:flow}, \ref{eqn:cGroupFlow}, \ref{eqn:cSocFinal}, \ref{eqn:power2}, \ref{eqn:cPower1}, \ref{eqn:cPOnPeak}, and \ref{eqn:cPOffPeak}.
\begin{equation}
	\begin{aligned}
		& \underset{\mathbf{y}}{\scalebox{1}{\text{min}}} \ \mathbf{r}^T\mathbf{y} \\
		\text{subject to}& \ \ \ \  \begin{bmatrix}
			\begin{bmatrix}A & 0 \end{bmatrix} \\
				D_{\text{eq}} \\
				P
				\end{bmatrix} \mathbf{y} = \begin{bmatrix}c_f \\ \mathbf{d}_{\text{eq}}\\\mathbf{p} \end{bmatrix}, \ \begin{bmatrix}
		\begin{bmatrix} B & 0 \end{bmatrix} \\
			D_{\text{ineq}} \\ 
			P_{\text{max}} \\
			P_{\text{on}} \\
			P_{\text{off}}
			\end{bmatrix}\mathbf{y} \le \begin{bmatrix}\mathbf{1} \\ \mathbf{d}_{\text{ineq}} \\ \mathbf{0} \\ \mathbf{0} \\ \mathbf{0} \end{bmatrix},
	\end{aligned}
\end{equation}


