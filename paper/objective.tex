\section{Objective Function}
The objective function in this work models the rate schedule used in \cite{rocky_mountain_power_rocky_2021}, where the cost is modeled as the monthly charge a transit authority receives from the power provider. The objective function includes charges for energy, power, and facility use and implements both on and off-peak rates.
\par The objective function also includes effects and costs of
uncontrolled loads. Uncontrolled loads might include the effects of
patrons charging personal electric vehicles, electric trains passing
through, CNG stations, etc. The loads used in this work were recorded
at the UTA Intermodal Hub station in Salt Lake City (SLC), Utah as the
average power sampled at uniform time intervals.
\subsection{Energy}
\par Energy cost is assessed per Kilowatt-hour of energy consumed and includes energy consumed by uncontrolled loads and bus chargers. Let $\mathbf{p}$ be the average external power used at each timestep, where $\mathbf{p}_i$ is the average power draw between $t_j$ and $t_{j + 1}$. The energy consumed by external loads from $t_j$ to $t_{j+1}$ is computed as 
\begin{equation}
	e^l_j = \mathbf{p}_i \cdot \Delta_t,
\end{equation}
where $\Delta_t$ is the change in time from $t_j$ to $t_{j+1}$ in hours. The energy consumed by bus chargers for the same interval is computed as  
\begin{equation}
	e^b_j = \sum_{k\in t}g_{i,k,l},	
\end{equation}
where $k\in t$ represents all values for $g$ that took place between $t_i$ and $t_{i+1}$ for every bus.
The total energy is computed as 
\begin{equation}\label{eqn:energy1}
	e_j = e^l_j + e^b_j.
\end{equation}
Equation (\ref{eqn:energy1}) can be written in standard form as 
\begin{equation}\label{eqn:energy2}
	\begin{aligned}
		e_j -\sum_{k\in t}g_{i,k,l} &= p_i \cdot \Delta_t \\
		\begin{bmatrix} 1_{e_j} & -1_{g_1} & \hdots -1_{g_n} \end{bmatrix} \begin{bmatrix}e_j \\ g_1 \\ \vdots \\ g_n \end{bmatrix} &= p_i \cdot \Delta_t.
	\end{aligned}
\end{equation}
\par Because power providers charge different rates for the total power consumed during the respective on and off-peak hours,  equation (\ref{eqn:energy2}) be modified to reflect the energy consumed in arbitrary time periods.  Let $T$ be a set of $t_j$, or just $j$, which will later be used to denote on and off-peak periods as $T_{\text{on}}$ and $T_{\text{off}}$. Equation \ref{eqn:energy2} can be expanded to compute the total energy consumed in $T$ as
\begin{equation}\label{eqn:energy3}
	\begin{aligned}
	e_T -\sum_{k\in T}g_{j,k,l} &= \left ( \sum_{j\in T}p_j \right ) \cdot \Delta_t \\
		\begin{bmatrix} 1_{e_T} & -1_{g_1} & \hdots -1_{g_n} \end{bmatrix} \begin{bmatrix}e_T \\ g_1 \\ \vdots \\ g_n \end{bmatrix} &= e^\text{load}_T.
	\end{aligned}
\end{equation}
\par For multiple time periods, the constraint can be expanded in matrix form, where row $i$ corresponds to the periods of time in $T_i$. Furthermore, by including the values for each $e_{T_i}$ in $\mathbf{y}$ and zero-padding appropriately, the expanded form of equation (\ref{eqn:energy3}) can be written as  
\begin{equation}
	E\mathbf{y} = \mathbf{e}^\text{load},
\end{equation}
where row $i$ in $E$ reflects equation (\ref{eqn:energy3}) for the time intervals in $T_i$, and $\mathbf{e}^\text{load}_i$ contains the energy consumed by uncontrolled loads during $T_i$.

\subsection{Power}
Power costs are computed for the maximum average power draw, where the average is computed over a 15 minute sliding window. The average power can be computed as the energy in the window divided by the window length in hours. In this case, a 15 minute window equates to a quarter hour. Let $\bar{p}_j$ be the average power from $j - 15$ to $j$. Equation (\ref{eqn:energy3}) can be adapted to compute the average power as
\begin{equation}\label{eqn:power1} 
	\begin{aligned}
		\bar{p}_j - \left ( \sum_{k\in T_j}\frac{1}{4}g_{i,k,l} \right ) &= \left ( \sum_{i\in T_j}p_i \right ) \cdot \frac{\Delta_t}{4} \\
		\begin{bmatrix} 1_{\bar{p}_j} & -\frac{1_{g_1}}{4} & \hdots -\frac{1_{g_n}}{4} \end{bmatrix} \begin{bmatrix}e_j \\ g_1 \\ \vdots \\ g_n \end{bmatrix} &= \frac{p_T \cdot \Delta_t}{4}.
	\end{aligned}
\end{equation}
Equation (\ref{eqn:power1}) can further be expanded and zero padded to compute the average power at each time, $t_j$ by applying equation (\ref{eqn:power1}) to the corresponding window as
\begin{equation}\label{eqn:power2}
	P\mathbf{y} = \mathbf{p}.	
\end{equation}
The maximum average power, denoted $\hat{p}$, is greater than or equal to each average power computed in equation (\ref{eqn:power2}).  This yields an additional set of inequality constraints 
\begin{equation}\label{eqn:cPower1}
	\begin{aligned}
		 \begin{bmatrix} 
			-1_{\hat{p}} & 1_{\bar{p}_0} & 0             & \hdots & 0 \\ 
	        	-1_{\hat{p}} & 0       & 1_{\bar{p}_1} & \hdots & 0\\
			-1_{\hat{p}} & 0       & 0 & \hdots    & 1_{\bar{p}_j} \\
		 \end{bmatrix}\mathbf{y} &\le \mathbf{0} \\ 
		 P_{\text{max}}\mathbf{y} &\le \mathbf{0}.
	\end{aligned}
\end{equation}
Because the max average power is minimized in the objective function, the value for $\hat{p}_{\text{max}}$ will be forced down to the value of the greatest average power computed in equation (\ref{eqn:power2}), and accurately reflect the maximum average power.
\subsection{On/Off Peak Rates}
Power providers divide each day into on and off-peak periods during which different rates are applied for both energy and power costs. Let $H$ and $L$ be the respective sets of all time indices in on and off peak periods respectively. The cost of energy during on-peak hours can be expressed as 
\begin{equation}\label{eqn:cEnergy1}
	\begin{aligned}
		c_{\text{energy}_H} &= \left ( \sum_{j\in H} e_j \right ) r_{e_\text{on}} \\
		&= \begin{bmatrix}r_{e_1} & 0 & \hdots & 0 & r_{e_4} & \hdots & 0 \end{bmatrix}\mathbf{y} \\
		&= \mathbf{r}_{e_\text{on}}^T\mathbf{y},
	\end{aligned}
\end{equation}
where $\mathbf{r}^\text{on}_{e}$ contains the value of $r^\text{on}_{e}$ at the index corresponding to $e_j$ in $\mathbf{y} \ \forall j \in H$. A similar formulation can be used to describe the cost of energy consumed during off-peak hours.  
\par An on-peak rate also applies to charges for power. Equation (\ref{eqn:cPower1}) can be adapted to only include rows that correspond to average power values during on-peak hours such that
\begin{equation}\label{eqn:cPOnPeak} 
	\begin{aligned}
		 \begin{bmatrix} 
			 -1_{\hat{p}_\text{on}} & 1_{\bar{p}_0} & 0             & \hdots & 0 \\ 
			 -1_{\hat{p}_\text{on}} & 0       & 1_{\bar{p}_1} & \hdots & 0\\
			 -1_{\hat{p}_\text{on}} & 0       & 0 & \hdots    & 1_{\bar{p}_j} \\
		 \end{bmatrix}\mathbf{y} &\le \mathbf{0} \\ 
		 P_{\text{on}}\mathbf{y} &\le \mathbf{0}.  
	\end{aligned}
\end{equation}
Similarly, the off-peak max average power can be computed as
\begin{equation}\label{eqn:cPOffPeak}
	\begin{aligned}
		 \begin{bmatrix} 
			 -1_{\hat{p}_\text{off}} & 1_{\bar{p}_0} & 0             & \hdots & 0 \\ 
			 -1_{\hat{p}_\text{off}} & 0       & 1_{\bar{p}_1} & \hdots & 0\\
			 -1_{\hat{p}_\text{off}} & 0       & 0 & \hdots    & 1_{\bar{p}_j} \\
		 \end{bmatrix}\mathbf{y} &\le \mathbf{0} \\ 
		 P_{\text{off}}\mathbf{y} &\le \mathbf{0},  
	\end{aligned}
\end{equation}
where each row corresponds to $\bar{p}_j \ \forall j \in L$.
\par Many power providers include a facilities charge.  The facilities
charge is assessed per kW of the maximum average power and ignores on
and off-peak times. The total max average power is calculated using equation (\ref{eqn:cPower1}).
\par The total power cost can be computed as the sum of the on-peak, off-peak, and facilities charges as 
\begin{equation}
	\begin{aligned}
		c_\text{power} &= \begin{bmatrix}r_{\hat{p}_{\text{on}}}&0 & \hdots & 0 & r_{\hat{p}_{\text{off}}} & 0& \hdots &0& r_{\hat{p}_{\text{facilities}}} \end{bmatrix}\mathbf{y} \\
			&= \mathbf{r}_{\hat{p}}^T\mathbf{y}.
	\end{aligned}
\end{equation}

\subsection{Objective Function}
The objective function combines the cost of energy and power, where the on-peak and off-peak energy is combined as 
\begin{equation}\label{eqn:cEnergy2}
	\begin{aligned}
		c_{\text{energy}} &= \mathbf{r}_{e_\text{on}}^T\mathbf{y} + \mathbf{r}_{e_\text{off}}^T\mathbf{y} \\
		&=\left ( \mathbf{r}_{e_\text{on}} + \mathbf{r}_{e_\text{off}} \right )^T \mathbf{y} \\
		&= \mathbf{r}_e^T\mathbf{y}.
	\end{aligned}
\end{equation}
The combined expression is given as 
\begin{equation}\label{eqn:objective}
	\begin{aligned}
		c_{\text{total}} &= c_{\text{power}} + c_{\text{energy}} \\ 
				 &= \mathbf{r}_e^T\mathbf{y} + \mathbf{r}_{\hat{p}}^T\mathbf{y} \\
				 &= \left ( \mathbf{r}_e + \mathbf{r}_{\hat{p}} \right )^T\mathbf{y} \\
				 &= \mathbf{r}^T\mathbf{y}.
	\end{aligned}
\end{equation}
\par Equation (\ref{eqn:objective}) is used as the objective function in a mixed integer linear program of the form
\begin{equation}
	\begin{aligned}
		& \underset{\mathbf{y}}{\scalebox{1}{\text{min}}} \ \mathbf{r}^T\mathbf{y} \ \text{subject to}\\
		& C_{\text{eq}}\mathbf{y} = \mathbf{c}_{\text{eq}}, \ C_{\text{ineq}}\mathbf{y} \le \mathbf{c}_{\text{ineq}},
	\end{aligned}
\end{equation}
where $C_{\text{eq}}, \mathbf{c}_{\text{eq}}, C_{\text{ineq}}$, and $\mathbf{c}_{\text{ineq}}$ are formed by stacking the equality and inequality constraints from equations (\ref{eqn:cFlow2}), (\ref{eqn:cGroupFlow2}), (\ref{eqn:cSocFinal}), (\ref{eqn:power2}), (\ref{eqn:cPower1}), (\ref{eqn:cPOnPeak}), and (\ref{eqn:cPOffPeak}),
\begin{equation}\label{eqn:finalObjective}
	\begin{matrix*}[c]
		\underset{\mathbf{y}}{\scalebox{1}{\text{min}}} \ \mathbf{r}^T\mathbf{y} \ \text{subject to}\\
		\begin{bmatrix*}[l]
				\tilde{A} \\
				D_{\text{eq}} \\
				P
				\end{bmatrix*} \mathbf{y} = \begin{bmatrix*}[l]\mathbf{c}_f \\ \mathbf{d}_{\text{eq}}\\\mathbf{p} \end{bmatrix*}, \ \begin{bmatrix*}[l]
			\tilde{B} \\
			D_{\text{ineq}} \\ 
			P_{\text{max}} \\
			P_{\text{on}} \\
			P_{\text{off}}
			\end{bmatrix*}\mathbf{y} \le \begin{bmatrix*}\mathbf{1} \\ \mathbf{d}_{\text{ineq}} \\ \mathbf{0} \\ \mathbf{0} \\ \mathbf{0} \end{bmatrix*}
	\end{matrix*}.
\end{equation}


