\section{Conclusions and Future Work }
In conclusion, the charge schedules developed in equation \ref{eqn:finalObjective} yield significant cost savings over the baseline case. These savings come from minimizing the average power consumption, and charging during off-peak hours. Cost savings are maintained in both uncontested and resource contrained scenarios.  There is also little to be gained by offering multiple charge rates because average power can be managed with high charge rates by reducing the charge duration. Furthermore, it was shown that when given the choice, the optimizer primarily selected high charge rates which reduces the solution to the feasible set of a single-rate formulation.
\par  Although multi-rate charging does not significantly reduce the monthly cost, it could be usefull in prolonging battery life. The high power rates observed in this work can reduce the lifespan of the battery, incentivising the use of low charge rates.  Therefore, future work might include a battery-centric scenario, which would lend itself well to a multi-rate charging framework.
\par Future work might also extend the discrete charge levels in this work to a continuous rate selection, and prepare the global plan for real-time deployment.  Because this work presents only a planning framework for a global solution over large stretches of time, it is computationally infeasible to recompute for unplanned events. Additional work might use a precomputed global plan in real-time execution, and manage stochastic events by returning to the global plan. The global plan could also be taylored for such an environment by optimising in a more robust manner, where areas of high uncertainty are generally avoided.
