\section{Fiscal Rate Schedule}
One objective of this work is to minimize the fiscal cost associated with power use and uses the Rocky Mountain Power schedule 8 billing rates. This billing schedule includes an on-peak power charge, facilities charge, and both on and off peak energy charges.
\par The facilities charge is computed by calculating the average power over a 15 minute period.  The facilities charge is based on the maximum average value over the course of the month.  This section of the bill charges \$4.81 per kW. The On-Peak Power Charge is similarly calculated but only includes average values from designated on-peak hours.  Rocky Mountain Power charges \$15.73 per kW for this value.
\par the facilities power can be formulated as a linear set of constraints that include power used at each timestep for charging buses and external loads.  This can be expressed as 
\begin{align*}
	c_i = \sum_j{g_{i,j}} + p_i
\end{align*}
Where $c_i$ represnets the 
\par  The energy charges are billed per kWh and charge for each unit of energy used.  There are two rates: 5.8282\textcent for energy consumed during on-peak hours, and 2.6316\textcent for off-peak hours.
general introduction with details for demand and consumption charge and how these relate to the end-of-the-month billing.

    a. external loads (contribution)
    b. consumption charge
        i. total energy in Kwh
	ii. on-peak vs off-peak
	iii. constraints for consumption charge
    c. demand charge
        i. average power in 15 minute window
	ii. on-peak vs facilities 
	iii. constraints for on-peak and facilities charges
    d. total cost breakdown
        i. Show how Rocky Mountain Power uses these and what the cost weighting is.
