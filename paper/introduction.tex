\section{Introduction}
Recent calls for a reduced carbon footprint have pushed transit authorities to adopt electric buses (EB). Conversion to EB reduces environmental impact as EB provide zero emissions and access to renewable energy \cite{poornesh_comparative_2020}. 
\par These benefits are possible because EB draw power from electrical infrastructure. The loads introduced by charging are substantial and can exceed the grid capacity \cite{stahleder_impact_2019}\cite{deb_impact_2017}\cite{boonraksa_impact_2019}, requiring prohibitively expensive upgrades. The cost of upgrading is reflected in the billing structure used by power providers and can make large-scale charging undesireable for consumers. 
\par One approach to reducing charge costs, is to defer premature upagrades by efficiently managing how buses charge. However, developing charge plans must consider a number of factors. For example, all buses must maintain a minimum charge level while adhering to route schedules. Batteries must also have sufficient charge time and share a limited number of chargers. The focus of this work is to find an optimal charge schedule which meets these requirements while minimising expenses from grid use. This problem is refered to hereafter as the `bus charge problem'.
\par \textcolor{red}{Because the number of fast-chargers is limited, a bus may only charge if a charger is available and must leave on route as scheduled. Furthermore, because batteries may require several hours for a full charge, they may not have time finish. Lower charge rates may also be used for less power draw which further delays charge times.} 
\par \textcolor{red}{At the end of operation hours, each bus is stored in the bus depot and connected to a `slow' charger.  Slow-charging is preferable when possible because it draws less instantaneous power and prolongs battery life \cite{houbbadi_optimal_2019}.  Furthermore, charging during off-hours offers each bus the opportunity to charge at any time with no contention for charging resources. The buses reside in the bus depot until the next morning when they depart for their first stop.}
\par \textcolor{red}{To maintain the bus battery state of charge in a nominal operations environment, there are several factors to consider including route schedules, on/off peak hours, additional grid loads, and the charge needs of individual buses. Finding a fiscally optimal charge schedule is refered to hereafter as the `bus charge problem' and is the focus of this paper.}
\par The remainder of this paper is organized as follows: Section II gives a description of previous work for solving the charge problem and Section III describes the directed graph framework used for describing the operations environment.  Section IV extends the content of III to multi-graph solutions, and Section V describes a set of linear constraints that describe the battery charge dynamics.  Sections VI and VII describe the rate schedule used for billing and how this is phrased as an objective function to minimize. Finally, Sections VIII, IX, and X briefly describe the optimization software used to solve the mixed integer linear program as developed in previous sections, present results, and describe future work.
