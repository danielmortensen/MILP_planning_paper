\section{Introduction}
Recent calls for a reduced carbon footprint have pushed transit authorities to adopt battery electric buses (BEBs). Conversion from diesel and CNG to BEB reduces environmental impact \cite{zhou_optimization_2018} as BEBs provide zero emissions and access to renewable energy \cite{poornesh_comparative_2020}. 
\par These benefits are possible because BEB draw power from electrical infrastructure. The loads introduced by charging are substantial and can exceed the grid capacity \cite{stahleder_impact_2019}\cite{deb_impact_2017}\cite{boonraksa_impact_2019}, requiring prohibitively expensive upgrades. The cost of upgrading is reflected in the billing structure used by power providers and can make large-scale charging undesireable for consumers. 
\par One approach to reducing charge costs is to defer upgrades by efficiently managing when and at what rates buses charge. However, developing charge plans must consider a number of factors. All buses must maintain a minimum charge level while adhering to route schedules. When charging, batteries must have sufficient charge time and share a limited number of chargers. All charging must also be done while other services draw power from the grid, acting as uncontrolled loads. The focus of this work is to find an optimal charge schedule which meets these requirements and minimizes the cost of grid use in the presence of uncontrolled loads. This problem is refered to hereafter as the `charge problem'.  
\par The remainder of this paper is organized as follows: Section II describes related work and Section III outlines a graph-based framework for modeling the operations environment. Section IV incorporates the problem constraints involving battery charge dynamics and section V extends the content of III to account for differences between day and night operations.  Sections VI translate the rate schedule used for billing and into an objective function. Finally, Sections VII and VIII present results, and describe future work.
