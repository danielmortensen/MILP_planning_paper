\section{Introduction}
Recent calls for a reduced carbon footprint have pushed transit authorities to adopt electric buses (EB). Conversion to EB reduces environmental impact as EB provide zero emissions and access to renewable energy \cite{poornesh_comparative_2020}. 
\par These benefits are possible because EB draw power from electrical infrastructure. The loads introduced by charging are substantial and can exceed the grid capacity \cite{stahleder_impact_2019}\cite{deb_impact_2017}\cite{boonraksa_impact_2019}, requiring prohibitively expensive upgrades. The cost of upgrading is reflected in the billing structure used by power providers and can make large-scale charging undesireable for consumers. 
\par One approach to reducing charge costs, is to defer premature upagrades by efficiently managing how buses charge. However, developing charge plans must consider a number of factors. All buses must maintain a minimum charge level while adhering to route schedules. When charging, batteries must have sufficient charge time and share a limited number of chargers. The focus of this work is to find an optimal charge schedule which meets these requirements while minimizing fiscal expenses from grid use. This problem is refered to hereafter as the `charge problem'.  
\par The remainder of this paper is organized as follows: Section II gives a description of previous work for solving the charge problem and Section III describes a graph-based framework for modeling the operations environment.  Section IV extends the content of III to account for differences between day and night operations, and Section V incorporates the problem constraints involving battery charge dynamics.  Sections VI and VII translate the rate schedule used for billing and into an objective function to minimize. Finally, Sections VIII, IX, and X briefly describe the optimization software used to solve the resulting mixed integer linear program developed in previous sections, presents results, and describes future work.
