\section{Introduction}
The goal of public transportation is to provide transit services using cost effective and environmentally sound methods. To better meet these needs, electric buses are becoming more common. Benefits include the use of renewable energy, reduced emissions, and fewer maintenance costs \cite{poornesh_comparative_2020}. However, there are also extended charge times, electrical expenses, increased loads on power infrastructure \cite{stahleder_impact_2019}, \cite{deb_impact_2017}, \cite{boonraksa_impact_2019}, and limited battery lifespans \cite{houbbadi_optimal_2019}.  
\par Because charging requires time, buses have limited availability, and there are limited charging resources, charge times must be scheduled with care. This work addresses the problem of charge scheduling and proposes a methodology with the following considerations: bus availabilty, external loads, fiscal expenses, day/night charging, and variable charge rates. To the best of our knowledge, this is the first time these methods have been included in the same framework. 
\par The remainder of this paper is organized as follows... ===insert stuff here===

