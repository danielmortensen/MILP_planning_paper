\section{Introduction}
Recent calls for a reduced carbon footprint have led transit authorities to adopt battery electric buses (BEBs). Replacing diesel and CNG buses with BEBs reduces environmental impact \cite{zhou_optimization_2018} as BEBs provide zero vehicle emissions and can access renewable energy sources \cite{poornesh_comparative_2020, Mahmoud2016}.

\par Charging BEBs draws power from electrical infrastructure. The combined effect of BEB charging with other necessary loads can exceed the capacity of local distribution circuits \cite{stahleder_impact_2019, deb_impact_2017, boonraksa_impact_2019}, leading to expensive infrastructure upgrades. Power providers pass the cost of upgrades on to customers.  Thus, the benefits of large-scale electrified busing seem appealing at first, but are only practical if infrastructural upgrades can be deferred or avoided altogether.

\par One approach to deferring or avoiding upgrades is to intentionally manage when and at what rates buses should charge. An optimal charge plan must account for a number of physical constraints and operational realities. For example, buses must exceed a minimum charge level while adhering to route schedules, batteries must have sufficient time to charge, and buses must share a limited number of chargers. The focus of this work is to find an optimal charge schedule which meets these requirements and minimizes the cost of electricity and grid impacts in the presence of other uncontrolled loads. This problem is referred to hereafter as the ``charge problem''.

\par The remainder of this paper is organized as follows: Section II describes prior related work and Section III outlines a graph-based framework for modeling the environment including buses, routes, chargers, and uncontrolled loads. Section IV incorporates the problem constraints involving battery charge dynamics and Section V extends the the graph framework to account for differences between day and night operations.  Section VI translates the rate schedule used for billing into an objective function. Finally, Sections VII and VIII present results and describe future work, respectively.
