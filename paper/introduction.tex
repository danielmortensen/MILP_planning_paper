\section{Introduction}
Recent calls for a reduced carbon footprint have led transit authorities to adopt battery electric buses (BEBs). Replacing diesel and CNG buses with BEBs reduces environmental impact \cite{zhou_optimization_2018} as BEBs provide zero vehicle emissions and can access renewable energy sources\cite{poornesh_comparative_2020}.

\par Charging BEBs draws power from electrical infrastructure. The combined effect of BEB charging with other necessary loads can exceed the capacity of local distribution circuits \cite{stahleder_impact_2019}\cite{deb_impact_2017}\cite{boonraksa_impact_2019}, leading to the necessity of making  expensive infrastructure upgrades. Power providers pass along to customers the cost of upgrades.  Thus, the benefits of large-scale adoption of electrified bussing seem appealing at first, but public acceptance of electrified bussing is practical only if costly upgrades to the infrastructure can be deferred or avoided altogether.
%This paper presents a method that optimizes the load due to BEB charging in the presence of other uncontrolled loads and minimizes an objective that direclty measures the customer's electricity costs.  Using the optimal charging schedule produced by our method, customer costs and grid impacts may be determined answering critical questions about operating costs and grid impacts.

\par One approach to deferring or avoiding upgrades is to intentionally managing the bus charging schedule including when and at what rates buses should charge. An optimal charge plan must account for a number of physical constraints and operational realities. For example, buses must exceed a minimum charge level while adhering to route schedules. When charging, batteries must be provided sufficient time to charge, and buses share a limited number of chargers.  Moreover, charging takes place while other services draw power from the grid.  Hereafter, these other loads are called ``uncontrolled loads''. The focus of this work is to find an optimal charge schedule which meets these requirements and minimizes the cost of electricity and grid impacts in the presence of uncontrolled loads. This problem is refered to hereafter as the ``charge problem''.

\par The remainder of this paper is organized as follows: Section II describes prior related work and Section III outlines a graph-based framework for modeling the environment including buses, routes, chargers, and uncontrolled loads. Section IV incorporates the problem constraints involving battery charge dynamics and Section V extends the the graph framework to account for differences between day and night operations.  Sections VI translate the rate schedule used for billing into an objective function. Finally, Sections VII and VIII present results and describe future work, respectively.
