\section{Introduction}
The goal of public transportation is to provide transit services using cost effective and environmentally sound methods. Most public transportation systems include bus fleets.  Each bus in the fleet is assigned to routes with schedules. During operation hours, each bus traverses its assigned route and is expected to arrive at each stop on time. When a bus is not on-route, it resides in the bus depot. Buses in the fleet are operational for most of the day, and require large amounts of gas as fuel. This results in increased emissions and contributes to the local carbon footprint.  
\par In an effort to reduce bus fleet emissions, many transit authorities have begun adopting electric buses which, in addition to zero emissions also offer fewer maintenance costs, and access to renewable energy\cite{poornesh_comparative_2020}.
\par Charging electric buses however, draws power and induces a load on electrical infrastructure, or `the grid'. This infrastructure is built to support the maximum draw at any one time and becomes more expensive to maintain as higher loads are induced. 
\par Because maintaining high-draw infrastructure is expensive, power companies discourage heavy loads by billing higher energy rates for high load, or `on-peak', hours.  They also bill for average power with `demand' and `facilities' charges.  Charging electric buses has already been shown to significantly impact power infrastructure \cite{stahleder_impact_2019}\cite{deb_impact_2017}\cite{boonraksa_impact_2019} and, if left unchecked, these additional loads could make large-scale additions cost-prohibitive. Hence, load management is critical to the continued expansion of electric vehicals.  
\par Another significant consideration for using electric buses is charge time. A patrolium engine can refuel in a matter of minutes, while an electric bus may require several hours. This extended refuel time is not present in traditional operations and requires that charge times be planned with care. 
\par Charge planning revolves around several constraints; namely route schedules, charger availability, and charge rate.  During the day, buses have access to a number of fast chargers.  Fast chargers quickly replenish bus batteries, but are limited in number and significantly load the grid.  During the night, each bus is connected to a depo charger which use slower charge rates, charge during non-peak hours, stress the grid less, and prolong battery life \cite{houbbadi_optimal_2019}.
\par The remainder of this paper is organized as follows... ===insert stuff here===

