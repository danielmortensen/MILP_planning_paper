\documentclass{article}
\usepackage{xcolor}
\usepackage[margin=0.5in]{geometry}
\usepackage{amsmath}
\usepackage{emoji}
\newcommand\formatfeedback[2]
{%
	\textbf{Reviewer:} \textcolor{red}{#1} 
	\leavevmode\\[0.1in] \textbf{Response:} \textcolor{blue}{#2}
}
\begin{document}
\noindent Dear Dr. Bart van Arem, \\ \\
My Colleages and I are grateful for the opportunity to revise our paper. The reviewers have provided excellent feedback which we have incorporated to improve the quality of the manuscript. Our response to each reviewer comment appears below. The reviewer comments are given in the red format and our responses are in blue.

\subsection*{Reviewer 2}
The comments for reviewer 1 are addressed here:
\begin{enumerate}
	\item \formatfeedback{The paper has been improved and the proposed modeling for the optimization problem is interesting. However, one of the key outstanding issue is that the authors spent no effort to cite/survey and critic several recent papers published in top-tier journals and addressing the same problem. Minimization of the cost of charging ``charging problem" with consideration of different rates of electricity prices such as time of use is not novel. By searching in the IEEE Xplore database with quote "Electric Bus fleets", there are several journal papers published in the area. Examples are:
\begin{itemize}
	\item DOI: 10.1109/JSYST.2021.3114271
	\item DOI: 10.1109/JSYST.2019.2926460
	\item DOI: 10.1109/ACCESS.2020.2964391
\end{itemize}
}{Thank you for the feedback \emoji{peach}. We have updated our literature review to include the recommended papers.}
	\item \formatfeedback{Each bus or group of buses have their own designated platform in a bus terminal. As such, it is very unlikely to assign buses for different chargers i.e., each bus will be assigned to a charger located in its platform. Hence, having the probability/possibility of  two buses (bus 1 and bus 2) to be connected to the same charger at time t is not applicable. usually, bus 1 and 2 will have different platforms and thus different chargers.}{There are certainly scenarios where each bus would have a separate platform. Our work was focused around what we observed at the Utah Transit Authority in Salt Lake City, where the station maintains two overhead fast chargers to be shared among buses during the day. We also anticipate that transit authorities will desire to save money by only installing the minimum number of chargers, which would lead to contention and a tendency to overbook. Our work focuses on the more general case, knowing that under different assumptions, the proposed method could be simplified.}
	\item \formatfeedback{Time steps will not be uniform in a bus fleet a they vary based on the arrival time(s) of buses, which also varies over the daytime.}{ We understand this comment to be asking why the proposed method doesn't align the time steps in the graph with the arrival and departure times of the buses. One contribution in the proposed work minimizes the cost of demand charge which is billed for the highest fifteen minute average power from both the uncontrolled loads and bus chargers. We use an evenly spaced sampling approach to establish a common temporal frame of reference so that the effects of power use from the uncontrolled loads can be combined with those of the electric chargers.} 
	\item \formatfeedback{Order of referencing within the text is not clear. It has to start from [1]....}{Good catch! I have updated the references so that they are numbered by appearance.}
	\item \formatfeedback{It is very strange talking about Fig. 5 before Fig. 4 in the text.}{We have updated the figures so that their numbers better follow their appearances.}
	\item \formatfeedback{In the reviewer's opinion, the authors complicated an optimization problem that could be easily formulated and solved via its reformulation as a graph. The main issue with such graph-based formulations is the scalability and computation complexity. The authors are advised to compare the proposed formulation with a regular optimization formulation under different sizes of bus fleets. There is little to no description about the superiority of the proposed graph-based technique in the results section.}{We understand the reviewer's comments to mean that because the proposed method was made more complicated when it was reformulated as a graph, it may not scale well and may not be worth the extra work. This technique minimizes the cost of electricity in the presence of uncontrolled loads. Because the uncontrolled loads are sampled at regular, discrete intervals, we needed a framework where effects from buses and other loads could be brought into temporal correspondance. Prior methods did not account for the uncontrolled loads in their solutions and as such, did not need to descritize their temporal components. This solution may be more complicated than other methods, but this is because the proposed method is more complete. We have shown that including the additional information in our method results in significant savings (see Fig. 17) and outperforms both a baseline and previous optimization technique. We have included an additional plot which shows how the number of constraints and variables increase the number of buses so that the reader may be aware of computational limitations. 
 }
\end{enumerate}
\subsection*{Review 3}
The comments for reviewer 3 are addressed here:
\begin{enumerate}
	\item  \formatfeedback{What is the benefit of using a graph-based network flow framework to present the BEB charging
optimization problem instead of an equation-based MILP mathematical formulation (considering
the charging variable as a decision)? Many similar published studies take the other approach, such
as (https://doi.org/10.1016/j.tre.2020.102056 and https://doi.org/10.1016/j.trd.2021.103009).}{How to respond to this? It feels like this is more a question of style... we just chose to solve the problem this way...  I suppose one main feature is to include the historical data, which was sampled discretely and thus fit well within a graph based framework.}
	\item \formatfeedback{Many other similar works related to BEB charging optimization are published in energy and
transportation journals (e.g., Transportation Research Journals group). A comparison between the
proposed and previous models is essential to highlight the paper’s contributions (better in a table).
Including charge demand, electricity ToU tariff, and day/night charging are not new.}{This comment seems to want us to compare our results with a previous model... but we did this in the results section.  Also, we never claimed to be the first to integrate demand, ToU, and day/night charging.  We did claim that this was the first time they had been integrated together with variable rate charging.}
	\item \formatfeedback{It is essential to illustrate the model assumptions (charging stations allocation, chargers’ power,
battery capacity, transit network) at the beginning of the model presentation (summarized). In
addition, it will be better to include a table for the value of all parameters.}{We did give the model assumptions in the beginning of the results section (the battery capacity is in Section A paragraph 6).}
	\item \formatfeedback{There is a constraint for the state of charge (SoC) to be greater than a lower threshold. Is there a
constraint for SoC to be less than a maximum threshold?}{Because we are using a CCCV model for the battery, the difference in state of charge decreases as the state of charge approaches its maximum.}
	\item \formatfeedback{In the literature, high-power chargers (fast chargers) are deployed to charge the BEBs during
operation. Will the results change if these chargers are used?}{It's possible that the results could change (in that perhaps there would be non-maximum optimal charge rates).  I wouldn't anticipate anything else change all that much...}
	\item \formatfeedback{The number of variables, constraints and the computing time of the solved models should be
presented in the paper to describe the complexity of the proposed model. In addition, the transit
network data and the charging system infrastructure configuration must be included.}{We can add a plot/table that gives the number of variables to help the reader understand the comlexity of the model.  As for the transit network data and charging system infrastructure configuration... Would it be sufficient for use to say that the algorithm doesn't take these into account and so we haven't simulated this?  Would it also be a good idea to talk about how we rely on Rocky Mountain Power to create tarrifs so that minimizing the tarrifs would resolve these sorts of issues? It seems like this all comes back to minimizig the demand charge (as each meter is given for each circuit)}
	\item \formatfeedback{In the text, Fig 5 before Fig 4 and Fig 13 before Fig 12.}{We have updated the figure references to reflect their use.}
	\item \formatfeedback{Equation 10 is equality or inequality?? The summation in Equation 39 is on buses or timeslots?}{Good catch, it was inequality.  We have updated equation 10. The summation in equation 39 was for all buses in a given timeslot. We have updated the wording to make this more clear.}
	\item \formatfeedback{In equation 11, the energy consumption $\delta_i$ is only related to the bus ID. How is it calculated in
the proposed model (what is the value)?}{The model itself does not compute $\delta_i$. We assume this as given (perhaps computed from historical data).}
\end{enumerate}
\end{document}
